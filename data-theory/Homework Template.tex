\documentclass[12pt]{article}

%% make references and citations clickable
\usepackage[backref,colorlinks=true, linkcolor=blue, citecolor=blue, urlcolor=blue, pdfborder={0 0 0}]{hyperref}

%% set the paper geometry
\usepackage[left=1in,top=1in,right=1in,bottom=1in,letterpaper]{geometry}

%% uncomment the next line if you need to present an algorithm
%\usepackage{algorithm,algorithmic}

%% standard AMS stuff
\usepackage{amssymb,amsmath}

%% for including urls by \url{url text}
\usepackage{url}

%% Some useful shorthand commands
\newcommand{\bfx}{{\bf x}}
\newcommand{\bfb}{{\bf b}}
\newcommand{\bfu}{{\bf u}}
\newcommand{\bfv}{{\bf v}}
\newcommand{\bfw}{{\bf w}}

\begin{document}

\title{Homework Assignment 1 -- Math 118, Fall 2020}
\date{[Insert Date Here]}
\author{[Insert your name here]}
\maketitle


%% -------------- Problem ------------------ 
\newpage
\subsection*{Problem 1} Suppose that $A\in\mathbb{R}^{m\times n}$, $\bfv\in\mathbb{R}^{n}$ and $B\in\mathbb{R}^{n\times r}$. Determine the number of floating point operations ({\em ie} additions,  multiplications and divisions) required to compute $A\bfv$ and $AB$.

\medskip
\noindent\textbf{Answer:} [Type your answer here. Make sure you clearly define all mathematical objects in the answer.]

%% -------------- Problem ------------------
\newpage
\subsection*{Problem 2. (10 points)} Let $\bfv = [1,0,5,-3]^{\top}$ and $\bfw = [-2,4,5,1]$.
\begin{enumerate}
	\item Compute $\|\bfv\|_1$, $\|\bfw\|_{3}$ and $\|\bfv\|_{\infty}$. \\
	\item Verify that $\langle\bfv,\bfw\rangle \leq \|\bfv\|_2\|\bfw\|_2$ by computing both sides. \\
\end{enumerate}

\medskip
\noindent\textbf{Answer:} [Type your answer here. Make sure you clearly define all mathematical objects in the answer.]

%% -------------- Problem ------------------
\newpage
\subsection*{Problem 3. (10 points)} Show that $\displaystyle \|\mathbf{v}\|_{p} := \left(\sum_{i=1}^{n} |v_i|^{p}\right)^{1/p}$ is a norm for $1\leq p \leq \infty$. {\em Hint: you will need to use H\"{o}lder's inequality:
$$
\sum_{i=1}|v_iw_i| \leq \|\bfv\|_{p}\|\bfw\|_{q} \text{ for all } \bfv, \bfw\in\mathbb{R}^{n} \text{ and } q:= 1-\frac{1}{p}
$$
}

\medskip
\noindent\textbf{Answer:} [Type your answer here. Make sure you clearly define all mathematical objects in the answer.]

%% -------------- Problem ------------------
\newpage
\subsection*{Problem 4. (10 points)} [Start a problem on a new page. Type the problem here, or give its reference.]

\medskip
\noindent\textbf{Answer:} [Type your answer here. Make sure you clearly define all mathematical objects in the answer.]

%% -------------- Problem ------------------
\newpage
\subsection*{Problem 5. (10 points)} [Start a problem on a new page. Type the problem here, or give its reference.]

\medskip
\noindent\textbf{Answer:} [Type your answer here. Make sure you clearly define all mathematical objects in the answer.]


\end{document}